\thispagestyle{empty}
\vspace*{36\baselineskip}
\hbox to \textwidth{\hrulefill}
\par
Ich versichere wahrheitsgemäß, die Arbeit selbstständig angefertigt, alle benutzten Hilfsmittel vollständig und genau angegeben und alles kenntlich gemacht zu haben, was aus Arbeiten anderer unverändert oder mit Abänderungen entnommen wurde.

Karlsruhe, den 29.03.2018 %\todo{date}
\vspace*{2\baselineskip}
\\

................................ \\
(Sachin Rajgopal)
\cleardoublepage

\vspace*{1em}
\begin{center}
	\textbf{Zusammenfassung}
\end{center}
\par
Eine Eigenschaft existierende Dokumentstandards ist, dass nur eine Benutzergruppe bezüglich des Sehvermögens (Sehende, Sehgeschädigte oder Blinde) bedient wird. Ein Standard, der es erlaubt alle drei Zielgruppen ein Dokument benutzen können, wäre somit besser für die Barrierefreiheit. 
Diese Arbeit präsentiert einen Ansatz mit barrierearmen EPUB 3 Dokumenten vor, wo sehende, sehgeschädigte und blinde Leser das gleiche EPUB 3 Dokument benutzen können. Dies passiert indem es eine „integrierte Umwandlungsfunktion“ gibt, die das „Aussehen“ des Dokuments verändert. Der neue Dokumentstandard wird auf zahlreichen Lesegeräten getestet, um die Kompatibilität zu überprüfen. Zusätzlich wird ein einfacher, Textverarbeitungsprogramm ähnlicher Editor präsentiert, sodass Benutzer barrierearme Dokument des neuen Standards ohne EPUB Fachwissen erstellen können. 


%\todo{Zusammenfassung (Deutsch)}
\cleardoublepage
\vspace*{1em}
\begin{center}
	\textbf{Abstract}
\end{center}
\par
Current document standards have the characteristic that they can only serve one group of users under the aspect of vision (sighted, visually impaired or blind reader). The logical conclusion of this would be a document standard that can serve all three groups would be for better accessibility.
This thesis presents a new approach of an universal accessible version of EPUB 3 documents, which will allow sighted, visually impaired and blind readers to use and share the same EPUB 3 document by an "integrated switching mechanism" to change the "look" of the document. The new document standard will be tested on several reading systems to examine how well it works. Furthermore, a simple editor, which will be similar to existing word processors, will be presented that allows users to easily create accessible EPUBs of the new document standard without knowing how an EPUB file is constructed or created. 

%\todo{Zusammenfassung (Englisch)}

\cleardoublepage

