\thispagestyle{empty}
\vspace*{36\baselineskip}
\hbox to \textwidth{\hrulefill}
\par
Ich versichere wahrheitsgemäß, die Arbeit selbstständig angefertigt, alle benutzten Hilfsmittel vollständig und genau angegeben und alles kenntlich gemacht zu haben, was aus Arbeiten anderer unverändert oder mit Abänderungen entnommen wurde.

Karlsruhe, den 29.03.2018 %\todo{date}
\vspace*{2\baselineskip}
\\

................................ \\
(Sachin Rajgopal)
\cleardoublepage

\vspace*{1em}
\begin{center}
	\textbf{Zusammenfassung}
\end{center}
\par
Eine Eigenschaft existierender Dokumentstandards ist, dass in Gänze nur eine Benutzergruppe bezüglich des Sehvermögens (Sehende, Sehgeschädigte oder Blinde) bedient wird. Ein Standard, der es erlaubt alle drei Zielgruppen mit einem Dokumentstandard zu bedienen, existiert noch nicht.
Diese Arbeit präsentiert einen Ansatz für barrierearme EPUB-3-Dokumente, die genau diese Vorgaben erfüllen. Also ein von sehenden, sehgeschädigten und blinden Lesern gleichwertig nutzbares Format. Dies passiert indem es eine „integrierte Umschaltfunktion“ gibt, die das „Aussehen“ des Dokuments verändert. Der neue Dokumentstandard wird auf zahlreichen Lesegeräten getestet, um die Kompatibilität zu überprüfen. Zusätzlich wird ein einfaches Textverarbeitungsprogramm präsentiert, sodass Benutzer barrierearme Dokumente des neuen Standards ohne EPUB Fachwissen erstellen können. 


%\todo{Zusammenfassung (Deutsch)}
\cleardoublepage
\vspace*{1em}
\begin{center}
	\textbf{Abstract}
\end{center}
\par
Current document standards have the characteristic that they can only serve one group of users under the category of vision (sighted, visually impaired or blind reader). Consequently a document standard that can serve all three groups would be better for accessibility.
This thesis presents a new approach of an universal accessible version of EPUB 3 documents, which will allow sighted, visually impaired and blind readers to use and share the same EPUB 3 document with an "integrated switching mechanism" to change the "look" of the document. The new document standard will be tested on several reading systems to examine how well it works. Furthermore, a simple editor, which will be similar to existing word processors, will be presented that allows users to easily create accessible EPUBs of the new document standard without knowing how an EPUB file is constructed or created. 

%\todo{Zusammenfassung (Englisch)}

\cleardoublepage

