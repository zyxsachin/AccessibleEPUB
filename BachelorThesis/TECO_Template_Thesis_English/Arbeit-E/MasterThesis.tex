\documentclass{wissdoc}

% Autor: Roland Bless 1996-2009, bless <at> kit.edu
% ----------------------------------------------------------------
% Diplomarbeit - Hauptdokument
% ----------------------------------------------------------------
%%
%% $Id: diplarb.tex 53 2009-12-10 12:23:37Z bless $
%%
% wissdoc Optionen: draft, relaxed, pdf --> siehe wissdoc.cls
% ------------------------------------------------------------------
% Weitere packages: (Dokumentation dazu durch "latex <package>.dtx")
%\usepackage{bibgerm}
%\usepackage[backend=biber]{biblatex} 
\usepackage{csquotes} 
\usepackage{tabularx}
\usepackage{booktabs}
\usepackage{multirow}
%\usepackage{tocbibind}
\usepackage{siunitx}
\usepackage{xcolor}
\usepackage{textcomp}
\usepackage{listings}
\usepackage{newfloat,caption}
\usepackage{subcaption}
\usepackage{footnote}
\usepackage{rotating}
\usepackage{pgfplots}
\usepackage{pgfplotstable}
\usepackage{url}
\usepackage{boxhandler}
\usepackage{tabu}
\usepackage{amssymb}
%\usepackage{subfig}
%\usepackage{subcaption}
%\usepackage{caption}
\usepackage{subcaption}
%\usepackage[plainpages=true]{hyperref}
\usepackage[space]{grffile}
%\usepackage[numbers,sort&compress]{natbib}
\usepackage[backend=bibtex,natbib=true,hyperref=true,doi=false,url=false]{biblatex}

\usepackage{graphicx}

%\usepackage{biblatex}
% \usepackage{varioref}
\usepackage{verbatim}
\usepackage{float}    %z.B. \floatstyle{ruled}\restylefloat{figure}
\usepackage[strings]{underscore}
%\usepackage[hidelinks]{hyperref}
% \usepackage{subfigure}
% \usepackage{fancybox} % für schattierte,ovale Boxen etc.
% \usepackage{tabularx} % automatische Spaltenbreite
% \usepackage{supertab} % mehrseitige Tabellen
% \usepackage[svnon,svnfoot]{svnver} % SVN Versionsinformation 
%% ---------------- end of usepackages -------------

%\svnversion{$Id: diplarb.tex 53 2009-12-10 12:23:37Z bless $} % In case that you want to include version information in the footer
%\hyphenation{if...-then...}
%% Informationen für die PDF-Datei
\pgfplotsset{compat=newest}

\hypersetup{
%%% styling of link inside pdf
	colorlinks,
  citecolor=black,
  filecolor=black,
  linkcolor=black,
  urlcolor=black,
%%%		
 pdfauthor={Sachin Rajgopal},
 pdftitle={Accessible EPUB}
 pdfsubject={Not set},
 pdfkeywords={Not set}
}
\DeclareFloatingEnvironment[fileext=frm,placement={!ht},name=Listing,within=section]{listing}

% Macros, nicht unbedingt notwendig
\input{macros}

 \renewcommand{\familydefault}{\sfdefault}
% Print URLs not in Typewriter Font
\def\UrlFont{\rm}

\newcommand{\specialcell}[2][c]{%
  \begin{tabular}[#1]{@{}c@{}}#2\end{tabular}}

\newcommand\todo[1]{\textcolor{red}{TODO: #1}}

\newcommand\hlcode[1]{\textcolor{red}{#1}}

\newcommand\citeable[1]{\textcolor{green}{\hl{citeable: #1}}}

\newcolumntype{$}{>{\global\let\currentrowstyle\relax}}
\newcolumntype{^}{>{\currentrowstyle}}
\newcommand{\rowstyle}[1]{\gdef\currentrowstyle{#1}%
  #1\ignorespaces
}

\newif\ifcomment
%\commenttrue %# Show comments


\newcommand{\blankpage}{% Leerseite ohne Seitennummer, nächste Seite rechts
 \clearpage{\pagestyle{empty}\cleardoublepage}
}

%% Einstellungen für das gesamte Dokument

% Trennhilfen
% Wichtig! 
% Im german-paket sind zusätzlich folgende Trennhinweise enthalten:
% "- = zusätzliche Trennstelle
% "| = Vermeidung von Ligaturen und mögliche Trennung (bsp: Schaf"|fell)
% "~ = Bindestrich an dem keine Trennung erlaubt ist (bsp: bergauf und "~ab)
% "= = Bindestrich bei dem Worte vor und dahinter getrennt werden dürfen
% "" = Trennstelle ohne Erzeugung eines Trennstrichs (bsp: und/""oder)

% Trennhinweise fuer Woerter hier beschreiben
\hyphenation{
% Pro-to-koll-in-stan-zen
% Ma-na-ge-ment  Netz-werk-ele-men-ten
% Netz-werk Netz-werk-re-ser-vie-rung
% Netz-werk-adap-ter Fein-ju-stier-ung
% Da-ten-strom-spe-zi-fi-ka-tion Pa-ket-rumpf
% Kon-troll-in-stanz
}
\lstset{
    frame=single,
    breaklines=true,
		basicstyle=\scriptsize,
	 escapeinside={\%*}{*)},
    %postbreak=\raisebox{0ex}[0ex][0ex]{\ensuremath{\color{red}\hookrightarrow\space}}
}

% Index-Datei öffnen
\ifnotdraft{\makeindex}
%%%%%%%%%%%%%% includeonly %%%%%%%%%%%%%%%%%%%
% Es werden nur die Teile eingebunden, die hier 
% aufgefuehrt sind!
\includeonly{%
titlepage,%
statement,% Ist in KA Pflicht für Diplomarbeiten
introduction,% Motivation, Zielsetzung, Gliederung
background,% Grundlagen 
%analysis,   % Problembeschreibung (Detail) und Related Work
%design,   % Beschreibung der Problemlösung (Konzepte, allg. Architektur, ...)
%implementation,  % Beschreibung der Umsetzung/Implementierung
%evaluation,      % Nachweis und Auswertung
epubDocumentStandard,
evaluationEpubStandard,
editor,
evaluationEditor,
futurework,% Future Work
summary  % Zusammenfassung der Ergebnisse 
}
\bibliography{Literature, Websites}
\bibliography{Accessible_EPUB}
\usepgfplotslibrary{groupplots}
\usetikzlibrary{pgfplots.groupplots}
%\addbibresource{diplarb.bib}

%%%%%%%%%%%%%%%%%%%%%%%%%%%%%%%%%%%%%%%%%%%%%%
\begin{document}

\frontmatter
\pagenumbering{roman}
\ifnotdraft{
 %% Titelseite
%% Vorlage $Id: titelseite.tex 54 2009-12-10 12:23:58Z bless $

\def\usesf{}
\let\usesf\sffamily % diese Zeile auskommentieren für normalen TeX Font

\newsavebox{\Erstgutachter}
\savebox{\Erstgutachter}{\usesf Prof.~Dr.~Rainer Stiefelhagen}
\newsavebox{\Zweitgutachter}
\savebox{\Zweitgutachter}{\usesf \todo{Eintragen}}

\begin{titlepage}
\setlength{\unitlength}{1pt}

\begin{picture}(0,0)(85,770)
\includegraphics[width=\paperwidth]{logos/KIT_Deckblatt}
\end{picture}

\vspace*{-45pt}\hspace*{400pt}\includegraphics[width=.1\paperwidth]{logos/SZS.png}

\thispagestyle{empty}

%\begin{titlepage}
%%\let\footnotesize\small \let\footnoterule\relax
\begin{center}
\hbox{}
\vfill
{\usesf
{\huge\bfseries Accessible EPUB
 \par}
\vskip 1.8cm
Bachelor Thesis\\
by\\[2mm]
\vskip 1cm

{\large\bfseries Sachin Rajgopal\\}
\vskip 1.2cm
Study Centre for the Visually Impaired Students (SZS)\\
Department of Informatics\\
%Universität Karlsruhe (TH)\\[2ex]
\vskip 3cm
\begin{tabular}{p{5.5cm}l}
First Reviewer: & \usebox{\Erstgutachter} \\
Second Reviewer: & \usebox{\Zweitgutachter} \\
Supervisor: & Dr. Thorsten Schwarz \\
\end{tabular}
\vskip 3cm
Project Period:\qquad 01/12/2017 -- 30/03/2018%XX/XX/20XX
}
\end{center}
\vfill
\end{titlepage}
%% Titelseite Ende


%%% Local Variables: 
%%% mode: latex
%%% TeX-master: "diplarb"
%%% End: 

 \blankpage % Leerseite auf Titelrückseite
 %
 % Die folgende Erklärung ist für Diplomarbeiten Pflicht
 % (siehe Prüfungsordnung), für Studienarbeiten nicht notwendig
 \thispagestyle{empty}
\vspace*{36\baselineskip}
\hbox to \textwidth{\hrulefill}
\par
Ich versichere wahrheitsgemäß, die Arbeit selbstständig angefertigt, alle benutzten Hilfsmittel vollständig und genau angegeben und alles kenntlich gemacht zu haben, was aus Arbeiten anderer unverändert oder mit Abänderungen entnommen wurde.

Karlsruhe, den 29.03.2018 %\todo{date}
\vspace*{2\baselineskip}
\\

................................ \\
(Sachin Rajgopal)
\cleardoublepage

\vspace*{1em}
\begin{center}
	\textbf{Zusammenfassung}
\end{center}
\par
Eine Eigenschaft existierende Dokumentstandards ist, dass nur eine Benutzergruppe bezüglich des Sehvermögens (Sehende, Sehgeschädigte oder Blinde) bedient wird. Ein Standard, der es erlaubt alle drei Zielgruppen ein Dokument benutzen können, wäre somit besser für die Barrierefreiheit. 
Diese Arbeit präsentiert einen Ansatz mit barrierearmen EPUB 3 Dokumenten vor, wo sehende, sehgeschädigte und blinde Leser das gleiche EPUB 3 Dokument benutzen können. Dies passiert indem es eine „integrierte Umwandlungsfunktion“ gibt, die das „Aussehen“ des Dokuments verändert. Der neue Dokumentstandard wird auf zahlreichen Lesegeräten getestet, um die Kompatibilität zu überprüfen. Zusätzlich wird ein einfacher, Textverarbeitungsprogramm ähnlicher Editor präsentiert, sodass Benutzer barrierearme Dokument des neuen Standards ohne EPUB Fachwissen erstellen können. 


%\todo{Zusammenfassung (Deutsch)}
\cleardoublepage
\vspace*{1em}
\begin{center}
	\textbf{Abstract}
\end{center}
\par
Current document standards have the characteristic that they can only serve one group of users under the aspect of vision (sighted, visually impaired or blind reader). The logical conclusion of this would be a document standard that can serve all three groups would be for better accessibility.
This thesis presents a new approach of an universal accessible version of EPUB 3 documents, which will allow sighted, visually impaired and blind readers to use and share the same EPUB 3 document by an "integrated switching mechanism" to change the "look" of the document. The new document standard will be tested on several reading systems to examine how well it works. Furthermore, a simple editor, which will be similar to existing word processors, will be presented that allows users to easily create accessible EPUBs of the new document standard without knowing how an EPUB file is constructed or created. 

%\todo{Zusammenfassung (Englisch)}

\cleardoublepage


 \blankpage % Leerseite auf Erklärungsrückseite
}
%
%% *************** Hier geht's ab ****************
%% ++++++++++++++++++++++++++++++++++++++++++
%% Verzeichnisse
%% ++++++++++++++++++++++++++++++++++++++++++
\ifnotdraft{
{\parskip 0pt\tableofcontents} % toc bitte einzeilig
\pagenumbering{roman}
%\cleardoublepage
%\addcontentsline{toc}{chapter}{\listfigurename}
%\listoffigures
%
%\cleardoublepage
%\addcontentsline{toc}{chapter}{\listtablename}
%\listoftables
%\addcontensline{toc}{section}{List of Tables}
%\pagenumbering{roman}
%\listoffigures
%\addcontensline{toc}{section}{List of Figures}
%\blankpage
%\listoffigures
%\blankpage
%\listoftables
%\blankpage
}
\cleardoublepage
\blankpage

%% ++++++++++++++++++++++++++++++++++++++++++
%% Hauptteil
%% ++++++++++++++++++++++++++++++++++++++++++
\graphicspath{{images/}}

\mainmatter
%\null
\newpage
\pagenumbering{arabic}
%% Einleitung.tex
%% $Id: einleitung.tex 28 2007-01-18 16:31:32Z bless $
%%

\chapter{Introduction}
\label{ch:Introduction}
%% ==============================
% CLEARLY SHOW CONTRIBUTIONS AND LINK THEM TO SECTIONS

%\section{Background}

\section{Motivation}

\begin{comment}
\textcolor{red}{To Edit}

Over the last several years, accessibility is becoming a word with ever increasing importance. Over the last several decades several nations have passed regulatory acts which guarantee equal treatment between all people\cite{webaim}. For example, in 1994 Germany passed the Accessibility Ordinance (Behindertengleichstellungsgesetz) which meant that no one can be disadvantaged due to their disability.

As a result of the regulatory acts in many nations it was required to have documents being able to be read by everyone. Everyone is supposed to have access to documents.

Making accessible documents in paper form is difficult, because each group needs to have its document printed separately. Blind people need to have documents printed in braille, which requires special equipment. Furthermore it is very heavy. Visually impaired people have document requirement which person to person, so it is very difficult to prepare printed documents for them.

Making documents accessible in electronic form is much easier, as the document can be adjusted to each group's requirements easily. The output is also adjustable so if a blind person prefers using a screen reader to a refreshable braille display they are free to do so with little to no extra effort. Visually impaired people can change font, size and colors with only a few clicks.

Currently, the dominant format for electronic documents are PDFs (Portable Document Format, extension \lstinline{.pdf}) and word documents(extensions \lstinline{.doc}, \lstinline{.docx}, \lstinline{.odt}). While PDF/UA (PDF/Universal Accessibility) has done much in terms of accessibility, there are still some shortcomings. First of all both formats have a predefined page size. While this is useful for printed documents, a computer screen can rarely display all contents of the document to the detriment of visually impaired people.\cite{EPUBzone} The font size is also predetermined and cannot be changed. While zooming can increase the apparent size of the font, the document width may not fit the screen. Furthermore semantic information normally is missing from the documents. For example, a PDF does not necessarily have the document language defined. A PDF might also not have a predefined reading order which means that the header and footer might be output by the screen reader every page. 

Conversely, this means that an electronic document format with no set document size, containing semantic and structure information and a set reading order would be better suited to meet the demands of accessibility. 

\textcolor{red}{To Edit End}

\end{comment}

In recent years, accessibility has become increasingly important, especially for accessible documents. Several nations have passed regulatory laws that ensure equal treatment of all people and ensure that documents are accessible to all \cite{webaim}.\\
The currently dominant format for accessible electronic documents are Microsoft Word and PDF (Portable Document Format) documents, or more precisely PDF/UA (PDF/Universal Accessibility) documents \cite{pdfua}. Both formats share several features, among them a predefined page size. While this is useful for printed documents, a computer screen can rarely display all contents of the document to the detriment of visually impaired people \cite{EPUBzone}. In figure \ref{fig:reflowablePDF} a PDF is shown both normally and in the reflowable reading mode of Adobe Acrobat. While the text now fits the size of the screen some formatting is lost. It is not exactly clear where paragraphs end and the font cannot be changed easily. Hypertext Markup Language (HTML) documents are another widespread format and they are accessible. However, images are not embedded in the document, so they have to be attached additionally. Consequently it is not possible to provide a document as a single file. A whole folder has to be sent with it, reducing its portability.
%,\textcolor{red}{EDIT} as shown in the example of figure \ref{fig:badPDF}, where the columns are only as long as part of the screen. 
Therefore, a document format without a fixed document size containing semantic and structural information and a fixed reading order would be better suited to meet the requirements of accessibility.


\begin{figure}[h]
	\begin{minipage}{0.5\textwidth}
		\centering
		\includegraphics[width=\linewidth]{figures/loremNormal.png}
	\end{minipage}\hfill
	\hspace{0.25cm}
	\begin{minipage}{0.5\textwidth}
		\centering
		\includegraphics[width=\linewidth]{figures/loremReflowable.png}
	\end{minipage}

	\caption{On the left: a PDF displayed normally. On the right: the same PDF using the reflowable reading mode}
	\label{fig:reflowablePDF}
\end{figure}
\begin{comment}
\begin{figure}[h]
	\includegraphics[width=\linewidth]{figures/badPDF.png}
	\caption{Example of a PDF not suitable for the visually impaired \protect\cite{szs}}
	\label{fig:badPDF}
\end{figure}
\end{comment}

Furthermore, different "selectable" forms of presentation would be advantageous, especially for graphics or mathematical formulas, such as having formulas in the LaTeX\footnote{https://www.latex-project.org//} source code for blind users or high-contrast images for users with limited residual vision. 
%This could be combined with EPUB 3 \cite{EPUBzone}.
%$\mbox{ }$

\section{EPUB}
EPUB stands for {\bf E}lectronic {\bf PUB}lication and is a format primarily used for books in an electronic format (E-book). The EPUB format was created by the International Digital Publishing Forum (IDPF) and the current version is 3.1 which is a minor update to EPUB 3\footnote{http://www.idpf.org/epub/31/spec/epub-changes.html} \cite{EPUBspecs}. EPUB uses Extensible Markup Language (XML) based formats like Extensible Hypertext Markup Language files (XHTML), an extension of HTML, and thus also uses the accessibility standards and guidelines already established in many nations like the Web Content Accessibility Guidelines (WCAG) \cite{WCAG}. Unlike HTML documents EPUB is a self contained format, meaning that images and other files are included in the format and are not separate like with HTML documents. XHTML allows content to adapt to different screen sizes and the EPUB content is therefore reflowable. Font type and size can also be adapted to the individual needs of the users. Visually impaired people could therefore adjust the document to their preferences in regards to font style, size and color. The EPUB 3 specification also contains guidelines for accessibility so these features are built in and not an afterthought \cite{EPUB3bp}. For example while EPUB document have reflowable content, they still support page numbers so that printed book users can refer to a page number and the EPUB user can find it.

%The EPUB working group has also made important changes from EPUB 2 to EPUB 3 which improve the accessibility of documents. For example, mathematical equations can now be displayed in MathML and there is better navigation and more support for Cascading Style Sheets (CSS). However, not all of these changes are yet supported by EPUB readers and devices \cite{EPUB30changes}. 


%EPUB stands for electronic publication and is a format primarily used for books in an electronic format (E-book). The EPUB format was created by the International Digital Publishing Forum (IDPF) and the current version is 3.1, which is a minor update to EPUB 3.\cite{EPUBspecs} EPUB uses XML based formats  like XHTML, and thus also uses the accessibility standards and guidelines already established in many nations like the Web Content Accessibility Guidelines (WCAG). \cite{WCAG} This was done as reading systems can have different screen sizes and the EPUB content can therefore be reflowable. Font type and size can also be changed. Visually impaired people could therefore adjust the document to their preferences. The EPUB 3 specification also contains guidelines for accessibility so these features are built in and not an afterthought.\cite{EPUB3bp}


\begin{figure}
	\includegraphics[width=\linewidth]{figures/epubFolderStructure.png}
	\caption{Folder structure within an EPUB file}
	\label{fig:epubFolderStructure}
\end{figure}

\subsection{Changes from EPUB 2 to EPUB 3 and EPUB 3.1}

The EPUB working group has also made some important changes from EPUB 2 to EPUB 3  to make it more accessible. Here is a short overview of the major changes in EPUB  which will be expanded upon \cite{EPUB30changes,EPUB31changes}:
\begin{itemize}
	\item Support for HTML5
	\item MathML
	\item Using a XHTML document for navigation instead of NCX
	\item Scripting
	\item CSS 3 support
\end{itemize}

Equations can now be displayed in MathML \cite{EPUB30changes}. In EPUB 2 they had to either be displayed in normal XHTML or as images. It is difficult to write math in normal XHTML, as complex symbols like fractions, roots and plus-minus might not be displayed properly. Images can display the equations properly, but only as a SVG file can they be scaled without quality loss. Furthermore, all images are not as accessible as MathML, which can be scaled properly and can be read by a screen reader.

In EPUB 2 the navigation was done with a  Navigation Control file for XML (NCX) file, usually named \lstinline|toc.ncx| \cite{EPUB3bp}. A NCX file is also written in XML, but the tags are different from XHTML. In EPUB 3 the navigation document is now done instead with a XHTML file, and can be created with regular XHTML elements like \lstinline|ol| and \lstinline|ul|, for ordered and unordered lists, respectively. XHTML is much more widely used and is already an important format in EPUB 3, while NCX files have a distinct syntax, like \lstinline|<navMap>| for tables, which has to be learned in addition to XHTML.  Replacing NCX means that the creator has one document format less to worry about.

Scripting with JavaScript was strongly discouraged in EPUB 2, but in EPUB 3 it is optional \cite{EPUB3bp}. This allows documents to be interactive, but it is recommended that information is not hidden if a device does not support scripting. This should be done with a process called progressive enhancement. In essence, content should first be designed in the most accessible way, and then layers of enhancement should be added, which might not be supported by all devices. For example, if the creator wants to add an animation with JavaScript to the document, it might be better to add several images showing the stills of the animation at important junctions. Each still image would also have alternative text for users with screen readers. The creator could then add an animation for devices that support it, enhancing their experience.

EPUB 2 supports Cascading Style Sheets (CSS) 2, while EPUB 3 supports CSS 2.1 with added modules from CSS 3 which were defined as a basic EPUB CSS Profile. CSS is used to describe the presentation of various files like HTML and XHTML. The EPUB CSS Profile was removed in EPUB 3.1 and the IDPF defined more general CSS support requirements in its place \cite{EPUB31changes}. Henceforth they also use the more general definition of CSS as mentioned by CSS Working Group, which results in creators not being required to learn specific CSS attributes only seen in EPUB.

One new feature in EPUB 3 are media overlays, which might be of considerable interest to people with accessibility requirements \cite{EPUB3bp}. This feature is supposed to help to integrate the Digital Accessible Information System (DAISY) (digital talking books) in EPUB 3. DAISY is a digital standard to create an audio substitute for printed media \cite{daisyAccessibility}. When it was created in 1994 the primary audio media were tapes and CDs, both with rather limited runtime. Discs with DAISY books, however, can hold up to 40 hours of audio, while the corresponding number for CDs is about 74 minutes \cite{wasIstDaisy}. DAISY books also allow the user to skip between chapters, pages or even sentences and create bookmarks. EPUB 3 has the same features, but media overlays now allow it to synchronize audio narration to text. The user can still use Text-To-Speech(TTS) rendering, but as it is computer generated, it might not pronounce all words properly. Prerecorded audio narration is of course better at this, and now the user can switch between reading and audio narration without navigating in an audio file.

However, many of these changes are not supported by reading devices, and this will be the topic of discussion in chapter \ref{ch:Evaluation EPUB Standard}, where reading devices are evaluated, both software and hardware \cite{EPUB30changes}.


\subsection{EPUB Structure}
What is EPUB exactly? It has the extension \lstinline{.epub}, but is actually just a renamed ZIP file (extension \lstinline{.zip}). ZIP is a archive format used to compress files. After changing the file extension, the new ZIP file can be decompressed and the files within the document can be accessed \cite{EPUB3bp}.

Compressing the folder back to a ZIP file has to be done with care. The mimetype should be the first file to added to the ZIP container and not the other two folders. If this is not done properly, the EPUB file is not in the proper format and some readers can not read the file.


\begin{figure}
	\begin{lstlisting}
	application/epub+zip
	\end{lstlisting}
	\caption{Contents of the \lstinline{mimetype} file}
	\label{fig:mimetype}
\end{figure}

The folder structure of an EPUB file is shown in figure~\ref{fig:epubFolderStructure}. The file named \lstinline{mimetype} has a single line indicating that the ZIP file contains an EPUB seen in figure \ref{fig:mimetype}. The META-INF folder contains just one file named \lstinline{container.xml}. This file has to indicate the location of the package file (extension \lstinline{.opf}), and is shown in figure \ref{fig:containerXML}. The name of the folder named OEBPS can actually be freely chosen, however, \lstinline{container.xml} has to describe the path to the package file correctly. OEBPS stands for Open eBook Publication Structure, which is the format superseded by EPUB \cite{OPSspecs}.

\begin{figure}
	\begin{lstlisting}
	<?xml version="1.0" encoding="UTF-8"?>
	<container xmlns="urn:oasis:names:tc:opendocument:xmlns:container" version="1.0">
	<rootfiles>
	<rootfile full-path="EPUB/package.opf" media-type="application/oebps-package+xml"/>
	</rootfiles>
	</container>
	
	\end{lstlisting}
	\caption{Contents of container.xml}
	\label{fig:containerXML}
\end{figure}

\subsubsection{Package document}

The OEBPS folder contains the package document which has the extension \lstinline{.opf}. The package document contains five XML declarations:

\begin{itemize}
	 \item \lstinline|metadata| (required)
\end{itemize}


The metadata, shown in figure \ref{fig:metadata}, which refers to data that provides information about other data\footnote{https://www.merriam-webster.com/dictionary/metadata}, declares information about the document in Dublin Core form \cite{dublinCore}. The Dublin Core Metadata Initiative is an organization managing terms used in metadata documents. Only \lstinline{dc:title, dc:language, dc:identifier} and one \lstinline{meta} are required, but there are several optional ones\footnote{http://dublincore.org/documents/dcmi-terms/}. The data entries have to follow the specifications. 

\begin{figure}
	\begin{lstlisting}
	<metadata xmlns:dc="http://purl.org/dc/elements/1.1/" xmlns:opf="http://www.idpf.org/2007/opf" xmlns:dcterms="http://purl.org/dc/terms/">
		<dc:title>BookTitle</dc:title>
		<dc:creator>Author</dc:creator>
		<dc:language>en</dc:language>
		<dc:publisher>Publisher</dc:publisher>
		<dc:identifier id="BookId">BookIdentificationNumber</dc:identifier>
		<meta property="dcterms:modified">2018-02-07T10:43:33Z</meta>
		<meta name="AccessibleEPUB" content="0.1.0" />
	</metadata>
	\end{lstlisting}
	\caption{The various Dublin Core tags in the \lstinline{metadata}}
	\label{fig:metadata}
\end{figure}

\begin{itemize}
	\item \lstinline{manifest} (required)
\end{itemize}

The manifest contains all of the files which should be included in the EPUB file with an ID, the path to each file and an appropriate declaration of the media type. A sample manifest is shown figure \ref{fig:manifest}  Properties also have to entered. If a XHTML file is a navigation document, \lstinline{nav} has to be entered in the properties.

\begin{figure}
	\begin{lstlisting}
	<manifest>
		<item id="ncx" href="toc.ncx" media-type="application/x-dtbncx+xml"/>
		<item id="Content.xhtml" href="Text/Content.xhtml" media-type="application/xhtml+xml"/>
		<item id="style.css" href="Styles/style.css" media-type="text/css"/>
		<item id="navid" href="Text/nav.xhtml" media-type="application/xhtml+xml" properties="nav"/>
	</manifest>
	\end{lstlisting}
	\caption{Files listed in the \lstinline{manifest}}
	\label{fig:manifest}
\end{figure}

\begin{itemize}
	\item \lstinline{spine} (required)
\end{itemize}

The spine declares the default reading order of the EPUB. The reader is free to go to whatever page they want, but the \lstinline|spine| allows the document creator to set a logical order of the files in the spine. The only file types which do not require a fallback in the \lstinline|spine| are XHTML and SVG files \cite{EPUB3bp}. A sample spine is shown in figure \ref{fig:spine}.

\begin{figure}
	\begin{lstlisting}
	<spine toc="ncx">
		<itemref idref="Content.xhtml"/>
	</spine>
	\end{lstlisting}
	\caption{The reading order declared in the \lstinline{spine}}
	\label{fig:spine}
\end{figure}
\begin{itemize}
	\item \lstinline{guide} (optional)
\end{itemize}
\begin{itemize}
	\item \lstinline{bindings}. (optional)
\end{itemize}
The guide is deprecated and only needed for EPUB 2, but it is useful for backwards compatibility. The bindings section is used for fallback options for interactive media. Both of these are optional and will therefore not be discussed further in this bachelor thesis.

\subsubsection{Navigation}

The OEBPS folder also contains a file, normally named nav.xhtml, which is the table of contents. The table of contents can be used to navigate to any tagged section and is created by simply making an ordered or unordered list in XHTML. A short example is shown in figure~\ref{fig:tableOfContents}. 

\begin{figure}
	\begin{lstlisting}
	<nav epub:type="toc" id="toc">
	<h1>Table of Contents</h1>
		<ol>
			<li>
			<a href="../Text/Content.xhtml">Start</a>
			</li>
		</ol>
	</nav>		
	\end{lstlisting}
	\caption{An example table of contents}
	\label{fig:tableOfContents}
\end{figure}

\subsubsection{Content documents}

One of the primary documents of an EPUB document are Extensible Hypertext Markup Language (XHTML) files. XHTML is used, because it is a web format and websites have to be displayed on a variety of devices, from phones with screen dimensions of a few inches to 50 inch television screens with varying display resolutions. The content is therefore reflowable and suits Ebook readers as they are available in various sizes. 

\begin{figure}[h]
	\begin{center}
		\includegraphics[width=\linewidth/2]{figures/bitmapVsSvg.png}
	\end{center}
	
	\caption{$Source: https://upload.wikimedia.org/wikipedia/commons/thumb/\\6/6b/Bitmap_VS_SVG.svg/1280px-Bitmap_VS_SVG.svg.png$}
	\label{fig:bitmapSvg}
\end{figure}

Another primary document of EPUB documents are Scalable Vector Graphics (SVG) files. They are based on XML and while image formats like JPEG and PNG are raster based with pixels, SVG files are vector based. This means that they retain their appearance at any size and do not get blurry at larger sizes than they created at, which can be seen in figure \ref{fig:bitmapSvg}.


\section{Goals of this thesis}

The goals of this thesis are:

\begin{enumerate}
	\item Create an accessible document standard based on EPUB 3 with the following features:
	\begin{enumerate}
		\item Have one document for all users: sighted, visually impaired and blind
		\item Ability to change between display styles for each group of users
		\item Support for MathML
		\item The font type changes depending on the user group.
	\end{enumerate}
	\item Develop an editor which will then be able to create EPUB files of this new standard. The main features will include:
		\begin{enumerate}
		\item Creating documents of the new accessible EPUB standard 
		\item Not requiring programming knowledge
		\item Inserting formulas using MathML using only LaTeX math terms
		\item Preview the display styles of each target group. 
	\end{enumerate}
\end{enumerate}

\begin{figure}
	\begin{center}
		\includegraphics[width=\linewidth/3]{figures/QuadraticEquation.png}
	\end{center}
	\caption{Quadratic Equation}
	\label{fig:quadEquaPng}
\end{figure}

\begin{figure}
	\begin{lstlisting}
	x = \frac{ - b \pm \sqrt {b^2 - 4ac}}{2a}
	\end{lstlisting}
	\caption{Quadratic Equation in LaTeX code}
	\label{fig:quadEquaLatex}
\end{figure}


\subsection{Document standard}
Existing literature, both electronic and in paper form, is frequently not accessible due to the figures and mathematical formulas in them. If the figure does not have proper captions or alternative text, blind people will be unable to extract the knowledge sighted people do from it. Blind people read formulas with specialized mathematical braille notation, such as Nemeth code in the USA and Marburger mathematical code in Germany \cite{augenbitWiki}. However, few books are printed in braille, as braille printers are uncommon and expensive and the resulting books tend to be very heavy. When the textbooks are in electronic form, the mathematical equations have to written as LaTeX code. The quadratic equation is shown in figure~\ref{fig:quadEquaPng}, while the corresponding LaTeX code is shown in figure~\ref{fig:quadEquaLatex}. Visually impaired people have different requirement for mathematics and graphics. They should be scalable and not get pixelated when the size in increased, as shown in figure \ref{fig:bitmapSvg}. This is best done with SVGs. Visually impaired people also need the text to be in a sans-serif font, like Helvetica or Arial \cite{pdfBarrierefrei}. They also require the font to be of a uniform size. As an example the exponent must the same font size as the variable itself, like x\textsuperscript{{\normalsize{2}}} instead of $x^{2}$. 


Sighted people have their formulas appear as a serif font as shown in figure \ref{fig:quadEquaPng}, visually impaired people need it to appear as sans-serif and blind people needs LaTeX code \cite{augenbitWiki}. This is difficult to do in one document, so separate documents have to be created for the blind and the visually impaired, creating additional work for the document creator. The new standard must include versions for sighted, visually impaired and blind people in one document to simplify the creation process. Instead of using the currently dominant formats, PDF and Microsoft Word, this will be done with EPUB 3, because it is more suited to electronic formats and allows dynamic content, thanks to CSS 3 and JavaScript. There will be a easy way to switch between the three versions. For example, when the blind version is chosen, content such as formulas will change to LaTeX code.

\subsection{Editor}
The second goal of this thesis is to develop an editor to create documents of this standard easily. It should not require programming ability and allow insertion of formulas and figures which will be available in all three versions. The editor should not be difficult to learn and be  What-You-See-Is-What-You-Get (WYSIWYG), much like a word processor, instead of XHTML which requires coding. It is intended to enable a guided creation of the documents.  % Einleitung
%% grundlagen.tex
%% $Id: grundlagen.tex 28 2007-01-18 16:31:32Z bless $
%%

\chapter{Background \& Related Work}
\label{ch:Background}
  % Grundlagen
\chapter{EPUB Document Standard}
\label{ch:EPUB Document Standard}

\section{Requirements}

One requirement of the new standard is that it incorporates a way to switch between three versions, them being:

\begin{itemize}
	\item Blind
	\item Visually Impaired
	\item Normal sighted
\end{itemize}

Switching between the three versions should be simple and clearly visible. It should not be hidden behind menus and settings and must be part of the document itself. A document of the standard must conform to EPUB 3 and not rely on external programs to switch between the versions.

In the following parts, the standard will be introduced and it will be discussed how key components of documents will be adapted to each group of visibility.


\subsection{Switching between versions}

Initially the intent was to find one new standard for universally accessible EPUB documents. Unfortunately, creating a single standard was not possible since not all EPUB readers support all EPUB 3 features yet.

\subsubsection{JavaScript}
JavaScript is used to make websites interactive, so it was the logical option to dynamically change the versions. The coding was also quite simple and only involved replacing a single line in the CSS file with programming. This was easy to do in JavaScript, and testing the individual XHTML files in the EPUB showed that it was successful. 

Shown in figure \ref{fig:jsStyleCss} are the contents of the file style.css. All it does is import the CSS rules of a different CSS file. The methods \lstinline|selectVisible()|, \lstinline|selectImpaired()| and \lstinline|selectBlind()| in \lstinline|script.js| all work the same way. To remember the version chosen, local and session storage are used. Both of them are used by web browsers to remember settings of websites. Session storage is saved until the web browser is closed, while local storage remains even after the browser has been closed. The methods check if local storage is available. If it is, then the method attempts to get the item \lstinline|accessibleEPUBcurrentCSS|. If it does not exist, it is created. This item is then set with the same import line in figure \ref{fig:jsStyleCss}, only with a different CSS filename for each version. If local storage is not available the same process is repeated with session storage. Local storage is preferable to session storage as changes are kept even after the program has been exited. This is not the case with session storage so the user has to set version again at the start of the program. Afterwards the method \lstinline|loadCSS()| is called which deletes the CSS rule of a XHTML file and adds the rule set by the earlier methods.

The content of the XHTML file responsible for the switching mechanism, \lstinline|VersionChanger.xhtml|, is shown in figure \ref{fig:js_switch}. It is important that both \lstinline|style.css| and \lstinline|script.js| are referred to, or the switching mechanism would be unsuccessful. In the body of the XHTML document, the method \lstinline{storageCSS()} is called whenever the XHTML file is loaded. This happens if the page is turned to that file. This method checks if local and then session storage are available. If either one is available, it then sets the item in the storage and calls \lstinline{loadCSS()}.

\begin{figure}
	
	\begin{lstlisting}
	@import url("../Styles/visible.css");
	\end{lstlisting}
	\caption{Contents of style.css}
	\label{fig:jsStyleCss}
\end{figure}

\begin{figure}
	
	\begin{lstlisting}
	<head>
	<title></title>
	<link rel="stylesheet" href="../Styles/style.css"/>
	<script src="../Misc/script.js"></script>
	</head>
	
	<body epub:type="frontmatter" onload="storageCSS();">
	<a id="a1" href="#a1" onclick="selectVisible();">Normal</a>
	<a id="a2" href="#a2" onclick="selectImpaired();">Visually impaired</a>
	<a id="a3" href="#a3" onclick="selectBlind();">Blind</a>
	</body>
	\end{lstlisting}
	\caption{The links in the JavaScript version}
	\label{fig:js_switch}
\end{figure}

There are some issues with the JavaScript version changer. If the software or hardware reader does not support temporary storage, the EPUB is always displayed in its default appearance. Due to this, switching with JavaScript is not a reliable option. As mentioned in chapter \ref{ch:Introduction}, JavaScript does not have to be supported by all EPUB readers. Nevertheless, the JavaScript implementation supports multiple XHTML files, because temporary storage is supported. For the same reason table of contents are also supported, as they are normally displayed in a separate file, named \lstinline|nav.xhtml|. 

\subsubsection{CSS}
The second method does not use JavaScript and uses some advanced features of CSS, introduced in CSS 3. This mainly refers to CSS selectors. Selectors allow HTML documents to have limited interactive capabilities, such as changing the appearance of a clicked element\footnote{https://www.w3schools.com/cssref/css\_selectors.asp} \cite{cssSelectors}.

Before showing how selectors work, the format of the content file should be discussed. It is shown in figure~\ref{fig:css_switch}. Unlike the JavaScript version, the links are in the same XHTML file and are shown at the beginning of an EPUB . In the JavaScript version, the links are on a separate page.


\begin{figure}
	
	\begin{lstlisting}
	<body>
	<a class="versionChanger" id="a1" href="#visible">Normal</a>
	<a class="versionChanger" id="a2" href="#impaired">Visually impaired</a>
	<a class="versionChanger" id="a3" href="#blind">Blind</a>
	
	<div style="padding:none" id="impaired" class="impaired">
	...
	</div>
	<div id="blind" class="blind">
	...
	</div>
	<div id="visible" class="visible">
	...
	</div>
	</body>
	\end{lstlisting}
	\caption{The links and divs in the CSS version}
	\label{fig:css_switch}
\end{figure}

The code shown in figure~\ref{fig:css_selector} is responsible for the switching mechanism. At first, only the visible version can be seen and the other two are hidden. The versions appear with the \lstinline|:target| selector, which affects the appearance of the element if it was a target of a link(\lstinline|<a>| element). If another link was clicked, then the current version becomes hidden again. However, the visible section would not become hidden, and this was a major problem. CSS selectors have limited capabilities compared to JavaScript, because they were intended to complement and not replace each other. After testing various CSS selectors, the \lstinline|~| (tilde) selector delivered the desired results. \lstinline|~| allows every element on the right hand side to be selected if it preceded by a left hand side element. So in the case of \lstinline|.impaired:target ~ .visible|, once the impaired link has been clicked, the visible section becomes hidden.

\vspace{1cm}
\begin{lstlisting}
.visible {
display:inline; 
}

.impaired:target ~ .visible {
display:none; 
}

.blind:target ~ .visible {
display:none; 
}

.impaired {
display:none; 
}

.blind {
display:none; 
}

.impaired:target{
display:inline; 
}

.blind:target{
display:inline; 
}
\end{lstlisting}
\vspace{-0.2cm}
\begin{figure}[H]

	\caption{The CSS selectors responsible for the dynamic version switching}
	\label{fig:css_selector}
\end{figure}

\subsection{Text}

\begin{figure}[H]
	
	\begin{center}
		\includegraphics[width=\linewidth/2]{figures/sansSerif.jpg}
	\end{center}
	
	
	\caption{Comparison of serif(left) and sans serif(right) fonts
		\\Source: https://cdncms.fonts.net/images/6bff0c2cdbbcca14/A.SerifSansPrint.jpg}
	\label{fig:sansSerif}
\end{figure}

The blind version does not need any special requirements for text formatting. The text size could be made much smaller, but leaving it at a standard size will allow other people to also read the text, such as for proofreading of the blind version.

The visually impaired version needs to use sans-serif fonts, as seen in figure \ref{fig:sansSerif}, instead of serif fonts, as it easier to read and identify individual characters with them \cite{pdfBarrierefrei}. Furthermore, the font should be larger than the normal version and the font size should be adaptable by the user. It is important that both superscripts and subscripts should be as large as regular text.



The font of the normal version can be chosen freely, but for this standard a serif font was chosen. Most importantly, the text should not go beyond the edge of the screen so that horizontal scrolling is not required. All normal text in the document will be between \lstinline|<p>| and \lstinline|</p>| tags. A comparison of the visually impaired and normal version can be seen in figures \ref{fig:viText} and \ref{fig:nText}. 

\begin{figure}[H]
	\centering
	\includegraphics[width=\linewidth*21/24]{figures/VItext.png}	
	\vspace{-0.3cm}
	\caption{Text of the visually impaired version \protect\cite{lipsum}}
	\label{fig:viText}
\end{figure}
\vspace{-0.5cm}
\begin{figure}[H]
	\centering
	\includegraphics[width=\linewidth*21/24]{figures/Ntext.png}	
	\vspace{-0.3cm}
	\caption{Text of the normal version \protect\cite{lipsum}}
	\label{fig:nText}
\end{figure}
\vspace{-0.5cm}
\subsection{Images}

All images need to have alternative text to describe them and preferably have a title given to them. This will help blind users as the screen reader will read out both of them. Usually the alternative text will be set as a property and the screen reader will have to identify the image. However, there was an issue. In some programs the screen reader does not read out the alternative text, so a different way had to be found to display it. The image will be part of a figure element of HTML5, shown in figure~\ref{fig:image_code}. A figure can also contain a caption, which will be identified by the screen reader as such, seen in  figure~\ref{fig:image_viimp} Furthermore, the alternative text will be inserted as a paragraph with the \lstinline|<p>| element. It will remain invisible in normal and visually impaired mode, but will appear as normal text in blind mode. The text should be surrounded by specific tags, like <Image> or <Graph> shown in figure~\ref{fig:image_blind}.

Images in visually impaired and normal mode will be displayed normally, but the image, caption and figure will all have a maximum width of 100\% of the screen size.


\begin{figure}[H]
	\lstset{language=HTML}
	\begin{lstlisting}
	<figure>
	<img title="KazirangaRhino" alt="Indian rhino standing in tall grass in Kaziranga National Park, India" src="..\Images\KazirangaRhino.png"/>
	<p class="transparent">
	&lt;Image&gt;Indian rhino standing in tall grass in Kaziranga National Park, India&lt;/Image&gt;
	</p>
	<figcaption> 
	Indian rhino in Kaziranga National park
	</figcaption>
	</figure>
	\end{lstlisting}
	\caption{Code of figure~\ref{fig:image_viimp} and \ref{fig:image_blind}}
	\label{fig:image_code}
\end{figure}

\begin{figure}[H]
	\centering
	\includegraphics[width=\linewidth]{figures/ImageVi.PNG}
	\caption{Image in 'Visual impairment' mode}
	\label{fig:image_viimp}
\end{figure}

\begin{figure}[H]
	\centering
	\includegraphics[width=\linewidth]{figures/ImageBl.PNG}
	\caption{Image in 'Blind' mode}
	\label{fig:image_blind}
\end{figure}






\subsection{Mathematical formulas}

Formulas should be stored in MathML. MathML is quite clunky and time consuming to write in, as seen in figure \ref{fig:math_code}, so a LaTeX to MathML converter is used. % For the sample documents, a online converting tool was used. The input formula was the quadratic equation, which appeared in \ref{fig:quadEquaPng}.



\begin{figure}[H]
	
	\lstset{language=HTML}
	\begin{lstlisting}
	<figure>
	<div role="math" class="math">
	<math xmlns="http://www.w3.org/1998/Math/MathML" id="Formula" title="Quadratic Formula" alttext="{x=\frac{-b\pm\sqrt{b^2-4ac}}{2a}}">
	<mstyle>
	<semantics>
	
	<mrow>
	<mi>x</mi>
	<mo>=</mo>
	<mfrac>
	<mrow>
	<mrow>
	<mo>-</mo>
	<mi>b</mi>
	</mrow>
	<mo>%*±</mo>*)
	<msqrt>
	<mrow>
	<msup>
	<mi>b</mi>
	<mn>2</mn>
	</msup>
	<mo>-</mo>
	<mrow>
	<mn>4</mn>
	<mo></mo>
	<mi>a</mi>
	<mo></ mo>
	<mi>c</mi>
	</mrow>
	</mrow>
	</msqrt>
	</mrow>
	<mrow>
	<mn>2</mn>
	<mo></mo>
	<mi>a</mi>
	</mrow>
	</mfrac>
	</mrow> 
	
	</semantics>
	</mstyle>
	</math>
	</div>
	
	<p class="transparent">
	$x = \frac{ - b \pm \sqrt {b^2 - 4ac}}{2a}$
	</p>
	
	</figure>
	\end{lstlisting}
	\caption{Code of figures \ref{fig:equation_normal} and \ref{fig:equation_blind}}
	\label{fig:math_code}
\end{figure}

\begin{figure}[H]
	\lstset{language=HTML}
	\begin{lstlisting}
	<mstyle scriptsizemultiplier="1" lspace="20%" rspace="20%" mathvariant="sans-serif">
	\end{lstlisting}
	\caption{Addition to the code shown in \ref{fig:math_code} for the visually impaired version to make the formula appear as sans serif and not in italics}
	\label{fig:math_vi_code}
\end{figure}


%\vspace{-0.6cm}




In the normal version the default display style is used, which uses a serif font and a combination of italicized and unitalicized letters, as shown in figure~\ref{fig:equation_normal}. Serif fonts are not suitable in the visually impaired version, so a sans serif had to be chosen. At first the font was changed in the CSS. This unfortunately resulted in the text not scaling properly, and therefore some characters were not properly spaced. The root symbol also wasn't displayed clearly. Instead the mstyle attribute of MathML had to be changed to sans serif which resulted in figure~\ref{fig:equation_viimp}. The mstyle attribute can not be changed in CSS, it has to be inserted into every math element. Figure \ref{fig:math_vi_code} shows the mstyle line which has to be included in the visually impaired section.

\begin{figure}[H]
	\centering
	\includegraphics[width=\linewidth*2/3]{figures/EquationNo.PNG}
	\caption{Equation in 'Normal' mode}
	\label{fig:equation_normal}
\end{figure}

\begin{figure}[H]
	\centering
	\includegraphics[width=\linewidth*2/3]{figures/EquationVi.PNG}
	\caption{Equation in 'Visual impairment' mode}
	\label{fig:equation_viimp}
\end{figure}


In the blind version, the text has to appear as LaTeX code. Much like the alternative text for images, it will appear as a separate paragraph enclosed in \lstinline{<p>}  tags. To signify that it is code, it has to be surrounded by \$ signs, as shown in figure~\ref{fig:equation_blind}. The math element is in a \lstinline{<div>} with CSS class "math", which is hidden in the blind version.

%\vspace{0.6cm}
\begin{figure}[H]
	\centering
	\includegraphics[width=\linewidth*2/3]{figures/EquationBl.PNG}
	\caption{Equation in 'Blind' mode}
	\label{fig:equation_blind}
\end{figure}




\chapter{Evaluation EPUB Standard}
\label{ch:Evaluation EPUB Standard}



\chapter{AccessibleEPUB Editor}
\label{ch:AccessibleEPUB Editor}

The AccessibleEPUB editor will be presented with each Windows form covering a separate section.

\section{Requirements}

There are a variety of requirements which have to be fulfilled the editor. Most importantly, it has easy to use and not require any programming knowledge to use, except for LaTeX code for equations. If no equations are needed in the document, then there must be programming ability asked of the user. 

\section{Programming language}

The first question was in which programming language should the editor be implemented in. C\# and Java were picked early as the two main options, as both of them are object oriented and natively support forms. Furthermore, both support the ability to make the programs themselves accessible for blind users. C\# has several accessibility properties, like AccessibilityDescription and AccessibilityName, which are passed to the screen reader. Java uses the Java Accessibility Bridge which makes it accessible to screen readers. Java is platform independent, and while C\# programs can run on Mac OS and Linux, it relies heavily on Windows and its features. However, Java is not already installed on any operating system, while C\# programs can run on Windows machine with only .NET as prerequisite. The target .NET version of the editor is contained in Windows 10. Therefore the editor was programmed in C\#, as the users were predominantly Windows users and don't have to install prerequisites.

\section{Main window with editor and preview}

\subsubsection{HTML Editor}

\begin{figure}
	\includegraphics[width=\linewidth]{figures/formJs.png}	
	\caption{Main window with HTML editor and JavaScript switching preview}
	\label{fig:formJs}
\end{figure}

The main window required the most programming and therefore also had its fair share of problems. The first issue was regarding the editor on the left hand side in figure \ref{fig:formJs}. At first a What-You-See-Is-What-You-Get(WYSIWYG) HTML editor was not intended, as it hides semantic information about the EPUB. The initial approach involved editing semantic information of XHTML elements, like images, and changing things like the src, alt, etc. This was still in the early, rough stages and the approach seemed too complicated, so the WYSIWYG editor was chosen instead. 

The second issue was how to implement the WYSIWYG HTML editor. Programming it from scratch would not be possible in the time allocated to a bachelor thesis, as it would involve creating a browser engine. Fortunately, the inbuilt browser in C\#, WebBrowser, allows editing with just a few lines of code. The WebBrowser control is based on Internet Explorer and displays web pages like it. It uses Internet Explorer version 6 as default, which does not display certain CSS properties such as \lstinline|max-width| and cannot display SVGs. To fix this issue, some code has to be run when the form has loaded which determines the Internet Explorer version on the computer and loads the newest one to the HTML editor. After this SVGs and the CSS properties functioned as desired. 

A major issue with the HTML editor is that content written in it is in HTML, while EPUB requires XHTML. While there aren't major differences, there are some small ones such as independent tags like \lstinline|<br>| have to be self closing and written as \lstinline|<br/>| in XHTML. If this is not done the EPUB reader will specify an error. Therefore a tool has to convert the HTML code to XHTML. There are several packages available, one of them being HtmlAgilityPack\footnote{http://html-agility-pack.net/}. It can convert HTML and XHTML and also correct HTML parsing errors such as not closed tags. HtmlAgilityPack functioned well and as expected at first. The resulting code was in XHTML. However, there soon was an error. Many Greek signs, such as the sign "$\wedge$", were displayed as question mark. This meant HtmlAgilityPack could not be used. 

Another available package was TidyManaged\footnote{https://github.com/markbeaton/TidyManaged}, was also used, but it did not convert the code properly to XHTMl. The third attempt involved using SgmlReader\footnote{https://github.com/lovettchris/SgmlReader}, which converts SGML content, like HTML, to XML content, like XHTML. It successfully parsed the HTML and was able to convert mathematical signs properly. It did not create a XML declaration at the top of the document, but this was simpler to solve. A standard XML declaration was just added to the document and the new XHTML document was properly rendered by the browser.


\begin{figure}
	\begin{center}
		\includegraphics[width=\linewidth]{figures/formCss.png}	
		\caption{Main window with HTML editor and CSS switching preview}
		\label{fig:formCss}
	\end{center}
\end{figure}

\subsubsection{Preview browser}

On the right hand size of the main window is the preview browser. It should display XHTML page in each of three versions(normal, visually impaired, blind). Since the HTML editor uses the WebBrowser control, it would have been easiest to use it on the right side too. However, Internet Explorer is unable to display MathML. Instead of showing the quadratic equation, as shown in figure \ref{fig:quadEquaPng}, it showed figure \ref{fig:IEmathml}. As a result, another browser had to be found. Only two browsers are able properly show MathML, Mozilla Firefox and Safari by Apple. The Firefox engine, Gecko, was chosen as it is open source and had a C\# browser package. After inserting an instance of a \lstinline|GeckoWebBrowser| in the form MathML was displayed properly.

\begin{figure}
	\begin{center}
		\includegraphics[width=\linewidth/2]{figures/IEmathml.png}	
		\caption{MathML depiction of the quadratic equation in Internet Explorer}
		\label{fig:IEmathml}
	\end{center}
\end{figure}

As seen in figures \ref{fig:formJs} and \ref{fig:formCss}, there is a small difference between the user interface of the JavaScript and CSS versions. The CSS version only uses one browser tab as preview, while the JavaScript version uses three. This is mentioned in chapter \ref{ch:EPUB Document Standard} and is due to the JavaScript version having a separate file named VersionChanger.xhtml, which handles switching versions. In the CSS version the whole content is in one XHTML file. Consequently, the CSS version is very easy to show in the preview. Only the three links shown in figure \ref{fig:css_switch} have to be added to HTML editor content and file is done.

This is much more complicated in the JavaScript version, since the default display style will always be shown. The CSS of the blind and visually impaired version had to be changed. The initial attempt to solve this problem was done by changing the CSS of each browser in a tab and accessing the \lstinline|GeckoStyleSheet| of a browser. Unfortunately, even after changing the \lstinline|GeckoStyleSheet| of a document, it still displayed the default one. An alternative approach consisted of accessing the head(\lstinline|<head>|) of the XHTML document and inserting the CSS there, but this too did not work. These two methods were preferable as they do not have significant input and output (IO) operations. Unfortunately, even slightly altered methods of the two did not function properly. As such an IO intensive approach had to be taken. The XHTML file with the content, \lstinline|Content.XHTML| had to be copied to two new files, \lstinline|BliContent.XHTML| and \lstinline|ImpContent.XHTML| for the blind and visually impaired version, respectively. Then the single line linking the style in XHTML head was replaced with one linking the corresponding CSS. 
Both files are then removed before saving the EPUB and then created again. This is additional overhead, but it was the method which succeeded.

\textcolor{red}{pandoc } %┬▒}
\includegraphics[width=2em]{figures/pandocSigns.png}

\chapter{Evaluation Editor}
\label{ch:Evaluation Editor}



%%% analyse.tex
%% $Id: analyse.tex 28 2007-01-18 16:31:32Z bless $

\chapter{Analysis}
\label{ch:Analysis}


     % Analyse
%%% entwurf.tex
%% $Id: entwurf.tex 28 2007-01-18 16:31:32Z bless $
%%
%% ==============================
\chapter{Design}
\label{ch:Design}


     % Entwurf
%\chapter{Implementation of your Project}
\label{ch:Implementation}    % Implementierung
%%% eval.tex
%% $Id: eval.tex 5 2005-10-10 20:55:48Z bless $

%%%%%%%%%%%%%
\chapter{Evaluation}
\label{ch:Evaluation}
%%%%%%%%%%%%%
        % Evaluierung
%% zusammenf.tex
%% $Id: zusammenf.tex 4 2005-10-10 20:51:21Z bless $
%%

\chapter{Conclusion and Future Work}
\label{ch:FutureWork}
%% ==============================
%
  % Future Work
\include{summary}   % Zusammenfassung und Ausblick

%% ++++++++++++++++++++++++++++++++++++++++++
%% Anhang
%% ++++++++++++++++++++++++++++++++++++++++++

\appendix
%\include{anhang_a}
%\include{anhang_b}

%% ++++++++++++++++++++++++++++++++++++++++++
%% Literatur
%% ++++++++++++++++++++++++++++++++++++++++++
%  mit dem Befehl \nocite werden auch nicht 
%  zitierte Referenzen abgedruckt
\cleardoublepage
\phantomsection
\addcontentsline{toc}{chapter}{\bibname}
%%
%\nocite{*} % nur angeben, wenn auch nicht im Text zitierte Quellen 
           % erscheinen sollen
%\bibliographystyle{itmabbrv} % mit abgekürzten Vornamen der Autoren
%\bibliographystyle{gerplain} % abbrvnat unsrtnat
% spezielle Zitierstile: Labels mit vier Buchstaben und Jahreszahl
%\bibliographystyle{itmalpha}  % ausgeschriebene Vornamen der Autoren

\printbibliography
%% ++++++++++++++++++++++++++++++++++++++++++
%% Index
%% ++++++++++++++++++++++++++++++++++++++++++
\ifnotdraft{
\cleardoublepage
\phantomsection
\printindex            % Index, Stichwortverzeichnis
}

\end{document}
%% end of file

